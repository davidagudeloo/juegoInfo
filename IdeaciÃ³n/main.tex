\documentclass{article}
\usepackage[utf8]{inputenc}
\usepackage[spanish]{babel}
\usepackage{listings}
\usepackage{graphicx}

\usepackage{hyperref}
\urlstyle{same}

\graphicspath{ {images/} }
\usepackage{cite}

\begin{document}

\begin{titlepage}
    \begin{center}
        \vspace*{1cm}
            
        \Huge
        \textbf{Ideación}
            
        \vspace{0.5cm}
        \LARGE
        Proyecto Final - Los primeros pasos
            
        \vspace{1.5cm}
            
        \textbf{David Agudelo Ochoa}
        
        \vspace{0.5cm}
        
        \textbf{José Manuel Rivera Villa}
            
        \vfill
            
        \vspace{0.8cm}
            
        \Large
        Despartamento de Ingeniería Electrónica y Telecomunicaciones\\
        Universidad de Antioquia\\
        Medellín\\
        Marzo de 2021
            
    \end{center}
\end{titlepage}

\tableofcontents
\newpage
\section{Ideas Iniciales}\label{Ideas}
Nombre del juego: ChangeSpace \newline Espacio: Cuadrícula de M*N (por el momento se tiene pensado 2 filas por 3 columnas) \newline Vista: Tipo plataforma 2D \newline Animación: Pixel art\newline Dinámica: La dinámica principal del mismo es la música. Cada nivel tendrá como espacio jugable la cuadrícula M*N completa.  El jugador será situado en la parte de debajo de la cuadrícula central. Luego, cada cierto tempo, he aquí la importancia de la música, se iluminarán un número x de zonas, advirtiendo al jugador que deberá evitar a toda costa las mismas. El objetivo será que el jugador se mueva de espacio en espacio, evitando estar situado en un espacio iluminado, pues tras un corto periodo de tiempo estos espacios se tornarán de un color sólido, y serán atacados con proyectiles con diversos tipos de trayectorias. Si el jugador se encuentra sobre alguno de ellos sus probabilidades de perder, serán extremadamente altas. La posición de los espacios sombreados se generará aleatoriamente y será al ritmo de la música que estará sonando. La dificultad estará dada por la cantidad x de zonas que se iluminarán con respecto a la cuadricula M*N, y la velocidad en la que se generen, nuevamente se resalta la importancia de la música. El jugador ganara puntos a medida que pase más tiempo sin perder, y el nivel se considerará completo una vez termine la canción. \newline Se planea incorporar distintas canciones con diferente tempo para adecuar la dificultad de cada nivel, en los cuales le corresponderá un distinto tamaño de cuadrícula de ser necesario.



\section{Explicación Visual}\label{Video}
Enlace al video: \newline \url{https://youtu.be/Rn6ekEqZTxw}


\bibliographystyle{IEEEtran}


\end{document}
